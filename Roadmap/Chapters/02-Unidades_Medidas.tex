\chapter{Unidades \& Medidas}

\section{Introducción}

La física se embarca en el desafío de comprender y explicar los fenómenos que observamos en la naturaleza. Su enfoque radica en proporcionar descripciones matemáticas que revelen las leyes y mecanismos detrás de la evolución de los fenómenos.

\section{Fenómeno Observado}

\begin{table}[h]
    \centering
    \renewcommand{\arraystretch}{1.5}
    \begin{tabular}{| m{3cm} | m{10cm} |}
        \hline
        \multicolumn{2}{|c|}{\textbf{Fenómenos Observados}} \\
        \hline
        \textbf{Teoría Física} & \begin{itemize}
            \item Conceptos, definiciones
            \item Principios, postulados
            \item Leyes y modelos
        \end{itemize} \\
        \hline
    \end{tabular}
    \caption{Tabla de Fenómenos Observados}
\end{table}

\subsection{Proceso de Medición}

\begin{table}[h]
    \centering
    \renewcommand{\arraystretch}{1.5}
    \begin{tabular}{|c|l|}
        \hline
        \multicolumn{2}{|c|}{\textbf{Procesos de Medición}} \\
        \hline
        \textbf{Elemento} & \textbf{Descripción} \\
        \hline
        \textbf{Número} & Un valor cuantitativo o magnitud \\
        \textbf{Con/Sin Dirección} & Indica si la medición es un vector (con dirección) o escalar (sin dirección) \\
        \textbf{Unidad} & La unidad de medida correspondiente \\
        \hline
    \end{tabular}
    \caption{Tabla de Procesos de Medición}
\end{table}

\newpage
\section{Sistema Internacional de Medidas}

\begin{table}[h]
    \centering
      \renewcommand{\arraystretch}{1.5}
    \begin{tabular}{|c|c|c|}
        \hline
        \multicolumn{3}{|c|}{\textbf{Unidades Básicas}} \\
        \hline
        \textbf{Cantidad Física} & \textbf{Nombre} & \textbf{Símbolo} \\
        \hline
        Longitud & Metro & m \\
        Masa & Kilogramo & kg \\
        Tiempo & Segundo & s \\
        Corriente eléctrica & Amperio & A \\
        Temperatura & Kelvin & K \\
        Cantidad de sustancia & Mol & mol \\
        Intensidad luminosa & Candela & cd \\
        Corriente & Ampere & A \\
        Velocidad & Metro por segundo & m/s \\
        Volumen & Metro cúbico & m^3 \\
        Fuerza & Newton & N \\
        Aceleración & Metro por segundo al cuadrado & m/s^2 \\
        Presión & Pascal & Pa \\
        Energía & Julio & J \\
        \hline
    \end{tabular}
    \caption{Tabla de Unidades Básicas}
\end{table}

\subsection{Sistema Britanico}
\begin{table}[h]
    \centering
    \renewcommand{\arraystretch}{1.5}
    \begin{tabular}{|c|c|c|}
        \hline
        \multicolumn{3}{|c|}{\textbf{Sistema Británico}} \\
        \hline
        \textbf{Nombre} & \textbf{Simbolo} & \textbf{Equivalencia} \\
        \hline
        Pulgada & in & 1 in = 2,54cm \\
        Onza & oz & 1 oz = 28,53g \\
        Galon & gal & 1 gal = 3,785L \\
        Libra & lb & 1 lb = 453,59g \\
        Pie & ft & 1 ft = 0,3048m \\
        \hline
    \end{tabular}
    \caption{Tabla de Sistema Británico}
\end{table}

\newpage
\section{Prefijos}

 Potencias de 10 que multiplican a la unidad básica.

\[
\text{mm} = 10^{-3} \, \text{m}
\]
Donde:
\begin{itemize}
    \item \( \text{mm} \): milímetro.
    \item \( \text{m} \): metro.
\end{itemize}

Esto significa que un milímetro es igual a la milésima parte de un metro.

\section*{Prefijos de Unidades}

\begin{table}[h!]
\centering
\begin{tabular}{|c|c|c|}
\hline
\textbf{Prefijo} & \textbf{Símbolo} & \textbf{Factor} \\ \hline
Giga & G & $10^9$ \\ \hline
Mega & M & $10^6$ \\ \hline
Kilo & K & $10^3$ \\ \hline
Mili & m & $10^{-3}$ \\ \hline
Micro & $\mu$ & $10^{-6}$ \\ \hline
Nano & n & $10^{-9}$ \\ \hline
\end{tabular}
\caption{Tabla de prefijos y factores correspondientes.}
\end{table}

\newpage
\section{Conversión de Unidades}

\textbf{Ejemplos de conversión de unidades}
\\[0.7cm]

\textbf{Ejemplo 1 :} 7 pies hacia metros.
\\
\textbf{Solución :}
\\$$
7 ft = 7ft (0,3048m/1ft) = 2,1336 \approx 2,13m
\\[0.7cm]

\textbf{Ejemplo 2 :} 70 kilometros por hora hacia metros por segundo.
\\
\textbf{Solución :}
\\$$
70 \frac{Km}{h} = 70 \frac{Km}{h} (\frac{1000m}{1Km})(\frac{1h}{3600s}) = 19,444... \frac{m}{s} \approx 19,44\frac{m}{s}
\\[0.7cm]

\textbf{Ejemplo 3 :} 10 metros por segundo hacia kilometros por hora.
\\
\textbf{Solución :}
\\$$
10 \frac{m}{s} = (\frac{1K}{1000m})(\frac{3600s}{1h}) = 36 \frac{Km}{h}



\newpage
\section{Densidad}
\subsection{Formula}
$$
Densidad = \frac{Masa}{Volumen}
\\[0.7cm]
\section{Conversión de Temperaturas}
\subsection{Kelvin}
\\
\[
K = ^\circ C + 273.15
\]
\\
\subsection{Fahrenheit}
\[
^\circ F = \left( ^\circ C \times \frac{9}{5} \right) + 32
\]


