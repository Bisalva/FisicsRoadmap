\chapter{Movimiento en el Plano}

En el sistema de coordenadas rectangular, los ejes $x$ y $y$ son independientes entre sí. 

\textbf{Por lo tanto:} Se puede estudiar por separado el movimiento en cada eje.

\subsection*{Movimiento de Proyectiles}

\begin{center}
    \begin{tikzpicture}[scale=1.2]
        % Ejes coordenados
        \draw[->] (-0.5,0) -- (5,0) node[below] {$x$};
        \draw[->] (0,-0.5) -- (0,4) node[left] {$y$};
        
        % Vector de velocidad inicial
        \draw[->, thick] (0,0) -- (3,3) node[right] {$V_0$};
        
        % Componentes del vector de velocidad inicial
        \draw[dashed] (3,3) -- (3,0) node[below] {$V_{ox}$};
        \draw[dashed] (3,3) -- (0,3) node[left] {$V_{oy}$};
        
        % Ángulo
        \draw (0.6,0) arc[start angle=0, end angle=45, radius=0.6];
        \node at (0.9,0.2) {$\theta$};
        
        % Gravedad
        \draw[->, thick] (4,3) -- (4,2) node[right] {$g$};
    \end{tikzpicture}
\end{center}

Las ecuaciones del movimiento son:

\[
V_{ox} = |V_0| \cos\theta
\]
\[
V_{oy} = |V_0| \sen\theta
\]

\[
a_x = 0, \quad a_y = -g
\]

Las ecuaciones de posición y velocidad:

\[
x = x_0 + V_{ox} \cdot t
\]
\[
y = y_0 + V_{oy} \cdot t - \frac{1}{2} g t^2
\]

\[
V_x = V_{ox}
\]
\[
V_y = V_{oy} - g t
\]

\subsection*{Ejemplo}

Se lanza una piedra desde el techo de un edificio de 40m con una velocidad de $10$ m/s en una dirección de $28^\circ$ respecto de la horizontal. Determinar:

\begin{enumerate}
    \item[a)] La altura máxima alcanzada.
    \item[b)] Su posición y velocidad 2s después de soltarse.
    \item[c)] Su alcance horizontal y su tiempo de vuelo.
    \item[d)] La velocidad de la piedra justo antes de tocar el suelo.
\end{enumerate}

\subsection*{Diagrama del Problema}

\begin{center}
    \begin{tikzpicture}[scale=1]
        % Edificio
        \draw[thick] (-0.5,0) -- (-0.5,4) -- (0.5,4) -- (0.5,0);
        \draw[thick, dashed] (0.5,4) -- (5,0); % Trayectoria
        
        % Coordenadas y vectores
        \draw[->] (-1,0) -- (6,0) node[below] {$x$};
        \draw[->] (0,-1) -- (0,5) node[left] {$y$};
        
        % Altura
        \draw[dashed] (-0.5,4) -- (-1,4);
        \node[left] at (-1,4) {$40m$};
        
        % Vector velocidad inicial
        \draw[->, thick] (0.5,4) -- (2.5,4.8) node[right] {$V_0 = 10m/s$};
        
        % Ángulo
        \draw (0.8,4) arc[start angle=0, end angle=28, radius=0.8];
        \node[right] at (1.1,4.2) {$28^\circ$};
    \end{tikzpicture}
\end{center}

\subsection*{Resolución}

\textbf{Condiciones iniciales:}

\[
t = 0, \quad x_0 = 0, \quad y_0 = 40m
\]

\[
V_{ox} = 10 \cos 28^\circ \approx 8.83 \quad \text{m/s}
\]

\[
V_{oy} = 10 \sen 28^\circ \approx 4.69 \quad \text{m/s}
\]

\subsection{a) Altura máxima}

En la altura máxima, la velocidad en $y$ es cero:

\[
V_y = 0 = (10 \sen 28^\circ) - 9.8t
\]

Despejando $t$:

\[
t = \frac{10 \sen 28^\circ}{9.8} \approx 0.479s
\]

Calculando la altura máxima:

\[
y = 40 + 10 \sen 28^\circ \cdot 0.479 - 4.9 (0.479)^2
\]

\[
y \approx 41.12m
\]

\subsection*{b) Posición y Velocidad a los 2 segundos}

\subsection*{Cálculo de la posición}

\[
x(2) = x_0 + V_{ox} \cdot t
\]

\[
x(2) = 0 + (10 \cos 28^\circ) \cdot 2
\]

\[
x(2) \approx 17.66 \text{ m}
\]

\[
y(2) = y_0 + V_{oy} \cdot t - \frac{1}{2} g t^2
\]

\[
y(2) = 40 + (10 \sen 28^\circ) \cdot 2 - 4.9 (2)^2
\]

\[
y(2) \approx 29.79 \text{ m}
\]

\subsection*{Cálculo de la velocidad}

\[
V_x = V_{ox} = 10 \cos 28^\circ \approx 8.83 \text{ m/s}
\]

\[
V_y = V_{oy} - g t
\]

\[
V_y = (10 \sen 28^\circ) - 9.8 (2)
\]

\[
V_y \approx -14.9 \text{ m/s}
\]

\subsection{c) Cuando la Piedra Toca el Suelo}

Se considera que cuando la piedra toca el suelo, su coordenada en $y$ es cero:

\[
y = 0
\]

\[
0 = 40 + (10 \sen 28^\circ) t - 4.9 t^2
\]

Esto es una ecuación cuadrática de la forma:

\[
at^2 + bt + c = 0
\]

Resolviendo con la fórmula general:

\[
t = \frac{-b \pm \sqrt{b^2 - 4ac}}{2a}
\]

Se obtiene:

\[
t_1 < 0, \quad t_2 \approx 3.38 \text{ s}
\]

\subsection*{Cálculo del alcance horizontal}

\[
t = \frac{x - x_0}{V_{ox}}
\]

Sustituyendo valores, se encuentra el alcance horizontal.


\[
y = y_0 + V_{oy} \left( \frac{x - x_0}{V_{ox}} \right) - \frac{g}{2} \left( \frac{x - x_0}{V_{ox}} \right)^2
\]

Esta ecuación tiene la forma general:

\[
y = C + A x - B x^2
\]

\section*{Cálculo del alcance horizontal}

El alcance horizontal se obtiene evaluando el tiempo total de vuelo en la ecuación de posición en $x$:

\[
x(3.38) = x_0 + V_{ox} \cdot t
\]

\[
x(3.38) = 0 + (10 \cos 28^\circ) \cdot 3.38
\]

\[
x(3.38) \approx 29.84 \text{ m}
\]

\section*{Dibujo de la trayectoria}

\begin{center}
\begin{tikzpicture}[scale=0.8]
    % Ejes coordenados
    \draw[->] (0,0) -- (7,0) node[right] {$x$};
    \draw[->] (0,0) -- (0,5) node[above] {$y$};
    
    % Trayectoria parabólica
    \draw[thick, domain=0:6, samples=100] plot (\x, {3 + 0.7*\x - 0.2*\x*\x});
    
    % Punto de lanzamiento
    \filldraw (0,3) circle (2pt) node[left] {$(x_0, y_0)$};
    
    % Punto de caída
    \filldraw (6,0) circle (2pt) node[below] {$(x_f, 0)$};
    
    % Vectores de velocidad
    \draw[->] (2,3.5) -- (3,4.5) node[right] {$\vec{V}$};
    \draw[->] (2,3.5) -- (3.5,3.5) node[below] {$V_x$};
    \draw[->] (2,3.5) -- (2,2.5) node[left] {$V_y$};
\end{tikzpicture}
\end{center}

\subsection{d) Velocidad en $t = 3.38$ s}

\subsection*{Componente horizontal}

\[
V_x = V_{ox} = 10 \cos 28^\circ \approx 8.83 \text{ m/s}
\]

\subsection*{Componente vertical}

\[
V_y = V_{oy} - g t
\]

\[
V_y = (10 \sen 28^\circ) - 9.8 \cdot 3.38
\]

\[
V_y \approx -22.43 \text{ m/s}
\]

\subsection*{Velocidad resultante y ángulo}

\[
\vec{V} = V_x \hat{i} + V_y \hat{j}
\]

\[
V = \sqrt{V_x^2 + V_y^2}
\]

\[
V = \sqrt{(8.83)^2 + (-22.43)^2}
\]

\[
V \approx 29.77 \text{ m/s}
\]

\[
\alpha = \tan^{-1} \left( \frac{|V_y|}{V_x} \right)
\]

\[
\alpha = \tan^{-1} \left( \frac{22.43}{8.83} \right)
\]

\[
\alpha \approx 72.75^\circ
\]




