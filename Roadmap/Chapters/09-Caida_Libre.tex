\chapter{Caída Libre}
\section{Suposiciones}

\begin{enumerate}
    \item Se desprecia la resistencia del aire.
    \item La aceleración de gravedad ``g'' no varía en el movimiento.
\end{enumerate}

\section*{Movimiento con aceleración constante}
\[
g = 9.8 \, m/s^2
\]
Dirección: hacia abajo.

\subsection*{En Particular}
Si la dirección en la cual se lanza un objeto es vertical, tendremos un movimiento unidimensional.

\section*{Ejemplo de un M.R.U.A}
\[
a_y = -g
\]

\subsection*{Posición y Velocidad}
\[
y(t) = y_0 + v_{0y}t - \frac{g}{2} t^2
\]

\[
V_y (t) = v_{0y} - g t
\]

\subsection*{Ejemplo}
Se lanza una piedra desde el borde del techo de un edificio de 40m de altura, con una velocidad de 10m/s hacia arriba. Determinar:

\vspace{1cm}

\begin{center}
\begin{picture}(100,100)
    % Dibujo del edificio
    \put(40,10){\line(0,1){80}} % Línea del edificio
    \put(40,90){\line(1,0){20}} % Techo
    \put(60,10){\line(0,1){80}} % Línea del otro lado
    \put(48,92){\circle*{5}} % Piedra lanzada
\end{picture}
\end{center}

\section*{Enunciado}
\begin{enumerate}
    \item[a)] La altura máxima.
    \item[b)] El instante en que la piedra alcanza su altura inicial.
    \item[c)] El instante en que la piedra alcanza una altura de 20 m sobre el suelo. Su velocidad en este instante.
    \item[d)] El tiempo que tomará la piedra en tocar el suelo (desde que fue lanzado). Su velocidad justo antes de tocar el suelo.
\end{enumerate}

\section*{Solución}

\subsection*{Datos Iniciales}
\begin{itemize}
    \item Se usa el sistema de referencia $S-R$.
    \item $t = 0$, comienza el movimiento.
    \item $y(0) = 40m$
    \item $V_y(0) = 10m/s$
\end{itemize}

\subsection*{a) Cálculo de la altura máxima}
En la altura máxima, la velocidad en la dirección $y$ es cero:
\[
V_y = 0
\]

\[
0 = 10 - 9.8t
\]

Despejando el tiempo $t$:

\[
t = \frac{10}{9.8} = \frac{50}{49} s
\]

\subsubsection*{Altura máxima}
\[
y = 40 + 10 \left(\frac{50}{49} \right) - 4.9 \left(\frac{50}{49} \right)^2
\]

\[
= \frac{2210}{49} \approx 45.1 m
\]

\begin{center}
\begin{picture}(100,100)
    % Dibujo del edificio y trayectoria
    \put(40,10){\line(0,1){80}} % Edificio
    \put(40,90){\line(1,0){20}} % Techo
    \put(60,10){\line(0,1){80}} % Otra línea del edificio
    \put(50,95){\vector(0,1){10}} % Flecha de subida
    \put(50,5){\vector(0,-1){5}} % Flecha de caída
    \put(55,100){$V_y = 0$}
    
    % Ejes coordenados
    \put(80,10){\vector(1,0){15}} % Eje X
    \put(80,10){\vector(0,1){15}} % Eje Y
    \put(96,8){$x$}
    \put(78,26){$y$}
\end{picture}
\end{center}

\section*{Datos Iniciales}
\begin{itemize}
    \item Aceleración debido a la gravedad: $g = 9.8 \text{ m/s}^2$
    \item Altura inicial: $y_0 = 40$ m
    \item Velocidad inicial en el eje $y$: $v_{0y} = 10$ m/s
\end{itemize}

\subsection{b) Momento en que el objeto alcanza su altura inicial }
Se usa la ecuación del movimiento:
\begin{equation}
    y = y_0 + v_{0y}t - \frac{g}{2} t^2
\end{equation}

Dado que $y = y_0$, tenemos:
\begin{equation}
    40 = 40 + 10t - 4.9t^2
\end{equation}
Restando 40 en ambos lados:
\begin{equation}
    0 = 10t - 4.9t^2
\end{equation}
Factorizando $t$:
\begin{equation}
    t(10 - 4.9t) = 0
\end{equation}
De aquí obtenemos dos soluciones:
\begin{equation}
    t = 0 \quad \text{ó} \quad t = \frac{10}{4.9} \approx 2.04 \text{ s}
\end{equation}
Por lo tanto, el objeto alcanza la altura inicial en $t \approx 2.04$ s.

\subsection{c)instante en que el objeto alcanza los 20m}

Se usa nuevamente la ecuación:
\begin{equation}
    y = y_0 + v_{0y}t - \frac{g}{2} t^2
\end{equation}
Sustituyendo $y = 20$:
\begin{equation}
    20 = 40 + 10t - 4.9t^2
\end{equation}
Restando 20 en ambos lados:
\begin{equation}
    0 = 20 + 10t - 4.9t^2
\end{equation}
Resolviendo la ecuación cuadrática:
\begin{equation}
    t = \frac{-10 \pm \sqrt{10^2 - 4(-4.9)(20)}}{2(-4.9)}
\end{equation}
Calculando el discriminante:
\begin{equation}
    t = \frac{-10 \pm \sqrt{100 + 392}}{-9.8}
\end{equation}
\begin{equation}
    t = \frac{-10 \pm \sqrt{492}}{-9.8}
\end{equation}
Aproximando:
\begin{equation}
    t_1 \approx -1.24 \text{ s}, \quad t_2 \approx 3.28 \text{ s}
\end{equation}
El tiempo válido es $t_2$, por lo que el objeto tarda aproximadamente **3.28 s** en llegar a la altura de **20 m**.

Dado que la velocidad a $t = 3.28s$ es:

\begin{equation}
V = 10 - 9.8 \times 3.28 \approx -22.14 \text{ m/s}
\end{equation}

\subsection*{d) Cuando el objeto toca el suelo}
Sabemos que $y = 0$, entonces:

\begin{equation}
0 = 40 + 10t - 4.9t^2
\end{equation}

Resolviendo la ecuación cuadrática:

\begin{equation}
t_1 \approx 4.05s, \quad t_2 \approx -2.013 \quad (\text{descartado por ser negativo})
\end{equation}

Por lo tanto, el objeto tarda aproximadamente $4.05s$ en tocar el suelo.

\subsection*{Velocidad justo antes de tocar el suelo}

\begin{equation}
V = 10 - 9.8 \times 4.05 \approx -29.69 \text{ m/s}
\end{equation}



